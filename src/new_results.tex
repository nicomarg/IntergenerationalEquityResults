\documentclass{article}
\usepackage[utf8]{inputenc}
\usepackage[T1]{fontenc}
\usepackage{lmodern}
\usepackage[english]{babel}
\usepackage{amsmath, amssymb, amsthm, stmaryrd}

\newcommand{\G}{\mathbb{G}}

\newtheorem{theorem}{Theorem}

\title{Extending results for an arbitrary set of generations}
\author{}

\begin{document}

\maketitle

\section{Notations}

We denote by $\G$ the set of generations, and $Y$ the set of utility values.\\
The set of utility streams is $X=Y^\G$.\\
We take $\kappa$ and $\mu$ two cardinal numbers such that $\kappa,\mu\leq|\G|$.

\section{Axioms}

We define the $\kappa$-relation as:
\[<_\kappa = \{(x,y)\in X^2|\forall i\in \G, x_i \leq y_i \land \kappa \leq
|\{j \in \G | x_j < y_j\}|\}\]
And $\leq_\kappa$ the reflexive closure of $<_\kappa$.\smallskip\par

\textbf{$\kappa$-Pareto}:
\[\forall x,y\in X, x <_\kappa y \Rightarrow x \prec y\]
We define the set of $\mu$-permutations on a set $E$ as the group generated by the permutations
which have a set of non-fixed points smaller than $\mu$:
\[\mu\mathfrak{S}_E = \langle\{\sigma\in\mathfrak{S}_E | |\{e\in E
|e\neq \sigma(e)\}|\leq \mu\}\rangle\]

\textbf{$\mu$-anonymity}:
\[\forall\sigma\in\mu\mathfrak{S}_\G,\forall x\in X, x \sim \sigma x\]

\section{Grading principle}
\ \par
\textbf{$\kappa$-$\mu$ grading principle}:
\[R_{\kappa\mu} = \{(x,y)\in X^2 | \exists\sigma\in\mu\mathfrak{S}_\G, x \leq_\kappa \sigma y\}\]

Adaptation of theorem 1 in \cite{svensson80}:
\begin{theorem}
    $R_{\kappa\mu}$ is a $\mu$-anonymous quasi-ordering on $X$, and $\kappa$-paretian if
    $\mu < \kappa$. %or $\mu\in\mathbb{N}$.
\end{theorem}
\begin{proof}
    \textbf{Reflexivity}: we can take $\sigma = Id$.\\
    \textbf{Transitivity}: if $x,y,z\in X$ such that $x R_{\kappa\mu} y$ and $y R_{\kappa\mu} z$,
    we take $\sigma,\sigma'\in \mu\mathfrak{S}_\G$ such that $x\leq_\kappa \sigma y$ and
    $y\leq_\kappa \sigma' z$. We have $x\leq_\kappa (\sigma\circ\sigma')z$,
    and $\sigma\circ\sigma'$ is a $\mu$-permutation as $\sigma$ and $\sigma'$ are.\\
    \textbf{$\mu$-anonymity}: if $\sigma\in\mu\mathfrak{S}_\G$
    then $\sigma^{-1}\in\mu\mathfrak{S}_\G$. Hence if $y=\sigma x$, as $\leq_\kappa$ is reflexive,
    $x\leq_\kappa\sigma^{-1}y$, thus $x R_{\kappa\mu} y$. We obtain $y R_{\kappa\mu} x$ the same
    way, which gives us $x I_{\kappa\mu} y$.\\
    \textbf{$\kappa$-Pareto}: if we have $x,y\in X$ with $x<_\kappa y$, is there a
    $\sigma\in \mu\mathfrak{S}_\G$ such that $y\leq_\kappa \sigma x$? If this is the case,
    we obtain $x<_\kappa \sigma x$.\\
    %If $\mu\in\mathbb{N}$, then [TODO].\\
    If $\mu < \kappa$, then if $K=\{g\in\G|x_g<x_{\sigma(g)}\}$ and
    $S = \{g\in\G|g \neq\sigma(g)\}$,
    by definition of $\kappa$, $\kappa\leq|K|$ and by definition of $\mu$, $\mu\geq |S|$.
    Hence $|K|>|S|$, thus there exists $g\in K\setminus S$, which means that
    $x_g<x_{\sigma(g)}$ and $g=\sigma(g)$, which is absurd.
\end{proof}

\section{Existence of SWF}

Adaptation of theorem 1 of \cite{basumitra03}:
\begin{theorem}
  There is no SWF satisfying $\kappa$-Pareto and $\mu$-anonymity axioms if $\kappa$ contains a subset $k$ of $\mathbb G$ such that $\mathbb G\setminus k$ is at least countable, $|\mathbb G|\geq |k|$ and $\mu$ contains permutations of size $|k|$.
\end{theorem}

Recall of the proof in \cite{basumitra03} : we say that $0$ and $1$ are in $Y$ and create for each real $0<z<1$ two utility streams $a(z)$ and $b(z)$ in $\{0,1\}^{\mathbb N}$ such that $b(z)$ is bigger than $a(z)$ by only one bit and that $a(z')$ is bigger than $a(z)$ by infinitely many bits when $z'>z$, so that we can swap 2 bits to compare $b(z')$ to $a(z)$ to say it is smaller and that all the $[W(a(z)),W(b(z))]$ are disjoint and non empty.

\begin{proof}
  We assume that $Y$ contains at least two elements that we identify wlog. to $0$ and $1$ (else there is only one possible utility stream).

  Let:
  \begin{itemize}
  \item $k\in\kappa$ satisfying the conditions
  \item $f$ be a surjection of $\mathbb G\setminus k$ in $\mathbb Q\,\cap\,]0,1[$
  \item $E(z)=\{g\in \mathbb G\setminus k\,|\, f(g)<z\}$ for $0<z<1$
  \item $\displaystyle a(z)=\left(\left\{\begin{array}{ll}1&\text{if }g\in E(z)\\0&\text{else}\end{array}\right.\right)_{g\in\mathbb G}$ for $0<z<1$
  \item $\displaystyle b(z)=\left(\left\{\begin{array}{ll}1&\text{if }g\in k\\a(z)&\text{else}\end{array}\right.\right)_{g\in\mathbb G}$ for $0<z<1$.
  \end{itemize}

  Therefore $\forall g\in k, a(z)_g=0$ and $b(z)_g=1$, so $a(z)<_\kappa b(z)$.

  Let $0<z<z'<1$.

  We are going to create two permutation $\sigma_1,\sigma_2\in\mu$ such as $\sigma_1 b(z)<_\kappa \sigma_2 a(z')$.

  If we have that $\forall 0<z<z'<1,|E(z')\setminus E(z)|\geq |k|$ (and I think we can construct $f$ so that it is verified if $|k|\leq|\mathbb G|$) then because $E(z')\setminus E(z)$ is infinite we can divide it in two sets $e_1$ and $e_2$ of same cardinality. Thenthere is an injection $\iota_1$ from $k$ to $e_1$ and $\iota_2$ from $k$ to $e_2$. Define $\sigma_i$ as $\underset{g\in k}{\text{\LARGE $\circ$}}\tau(g,(\iota_i(g))$.

  Let $g\in\mathbb G$.

  If $g\in k$ then
  \[(\sigma_1 b(z))_g=b(z)_{\underset{\notin E(z)\cup k}{\underbrace{\sigma_1^{-1}(g)}}}=0<1=(\sigma_2 a(z'))_g=a(z')_{\underset{\in e_2}{\underbrace{\sigma_2^{-1}(g)}}}.\]

  If $g\in e_1$ then
  \[(\sigma_1 b(z))_g=b(z)_{\underset{\in k}{\underbrace{\sigma_1^{-1}(g)}}}=1=(\sigma_2 a(z'))_g=a(z')_{\underset{=g\in e_1}{\underbrace{\sigma_2^{-1}(g)}}}.\]

  If $g\in e_2$ then
  \[(\sigma_1 b(z))_g=b(z)_{\underset{=g \notin E(z)\cup k}{\underbrace{\sigma_1^{-1}(g)}}}=0=(\sigma_2 a(z'))_g=a(z')_{\underset{\in k}{\underbrace{\sigma_2^{-1}(g)}}}.\]

  Else
  \[(\sigma_1 b(z))_g=a(z)_g\leq (\sigma_2 a(z'))_g=a(z')_g.\]

  Hence $\sigma_1b(z)<_\kappa \sigma_2a(z')$, so by $\kappa$-Pareto $W(\sigma_1b(z))<W(\sigma_2a(z'))$.

  Furthermore, since $\sigma_1,\sigma_2\in\mu$, by $\mu$-anonymity we have $W(\sigma_1b(z))=W(b(z))$ and $W(\sigma_2a(z'))=W(a(z'))$.

  So $W(b(z))<W(a(z'))$, so $[W(a(z)),W(b(z))]$ and $[W(a(z')),W(b(z'))]$ are disjoint and non empty (since $a(z)<_\kappa b(z)$), so to each real we can associate a unique non empty disjoint interval. That is absurd because those intervals would be countable.

\end{proof}



\bibliographystyle{alpha}
\bibliography{bibliography}
\end{document}

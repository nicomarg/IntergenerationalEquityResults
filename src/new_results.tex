\documentclass{article}
\usepackage[utf8]{inputenc}
\usepackage[T1]{fontenc}
\usepackage{lmodern}
\usepackage[english]{babel}
\usepackage{amsmath, amssymb, amsthm, stmaryrd}

\newcommand{\G}{\mathbb{G}}

\newtheorem{theorem}{Theorem}

\title{Extending results for an arbitrary set of generations}
\author{}

\begin{document}

\maketitle

\section{Notations}

We denote by $\G$ the set of generations, and $Y$ the set of utility values.\\
The set of utility streams is $X=Y^\G$.\\
We take $\kappa$ and $\mu$ two cardinal numbers such that $\kappa,\mu\leq|\G|$.

\section{Axioms}

We define the $\kappa$-relation as:
\[<_\kappa = \{(x,y)\in X^2|\forall i\in \G, x_i \leq y_i \land \kappa \leq
|\{j \in \G | x_j < y_j\}|\}\]
And $\leq_\kappa$ the reflexive closure of $<_\kappa$.\smallskip\par

\textbf{$\kappa$-Pareto}:
\[\forall x,y\in X, x <_\kappa y \Rightarrow x \prec y\]
We define the set of $\mu$-permutations on a set $E$ as the group generated by the permutations
which have a set of non-fixed points smaller than $\mu$:
\[\mu\mathfrak{S}_E = \langle\{\sigma\in\mathfrak{S}_E | |\{e\in E
|e\neq \sigma(e)\}|\leq \mu\}\rangle\]

\textbf{$\mu$-anonymity}:
\[\forall\sigma\in\mu\mathfrak{S}_\G,\forall x\in X, x \sim \sigma x\]

\section{Grading principle}
\ \par
\textbf{$\kappa$-$\mu$ grading principle}:
\[R_{\kappa\mu} = \{(x,y)\in X^2 | \exists\sigma\in\mu\mathfrak{S}_\G, x \leq_\kappa \sigma y\}\]

Adaptation of theorem 1 in \cite{svensson80}:
\begin{theorem}
    $R_{\kappa\mu}$ is a $\mu$-anonymous quasi-ordering on $X$, and $\kappa$-paretian if
    $\mu < \kappa$. %or $\mu\in\mathbb{N}$.
\end{theorem}
\begin{proof}
    \textbf{Reflexivity}: we can take $\sigma = Id$.\\
    \textbf{Transitivity}: if $x,y,z\in X$ such that $x R_{\kappa\mu} y$ and $y R_{\kappa\mu} z$,
    we take $\sigma,\sigma'\in \mu\mathfrak{S}_\G$ such that $x\leq_\kappa \sigma y$ and
    $y\leq_\kappa \sigma' z$. We have $x\leq_\kappa (\sigma\circ\sigma')z$,
    and $\sigma\circ\sigma'$ is a $\mu$-permutation as $\sigma$ and $\sigma'$ are.\\
    \textbf{$\mu$-anonymity}: \\
    \textbf{$\kappa$-Pareto}: if we have $x,y\in X$ with $x<_\kappa y$, is there a
    $\sigma\in \mu\mathfrak{S}_\G$ such that $y\leq_\kappa \sigma x$? If this is the case,
    we obtain $x<_\kappa \sigma x$.\\
    %If $\mu\in\mathbb{N}$, then [TODO].\\
    If $\mu < \kappa$, then if $K=\{g\in\G|x_g<x_{\sigma(g)}\}$ and
    $S = \{g\in\G|g \neq\sigma(g)\}$,
    by definition of $\kappa$, $\kappa\leq|K|$ and by definition of $\mu$, $\mu\geq |S|$.
    Hence $|K|>|S|$, thus there exists $g\in K\setminus S$, which means that
    $x_g<x_{\sigma(g)}$ and $g=\sigma(g)$, which is absurd.
\end{proof}

\bibliographystyle{alpha}
\bibliography{bibliography}
\end{document}

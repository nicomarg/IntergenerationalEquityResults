\documentclass{article}
\usepackage[utf8]{inputenc}
\usepackage[T1]{fontenc}
\usepackage{lmodern}
\usepackage[english]{babel}
\usepackage{amsmath, amssymb, amsthm, stmaryrd}

\title{Summary of the results seen in the Intergenerational Equity workshop}
\author{}
\date{}

\newtheorem{definition}{Definition}
\newtheorem{theorem}{Theorem}
\newtheorem{lemma}{Lemma}

\begin{document}

\maketitle

\section{Notations and definitions}

We denote by $Y$ the set of utility values, and $X=Y^{\mathbb{N}}$ the set of
utility streams.\smallskip\\
Given a utility stream $x$ and an integer $i$, $x_i$ is the $i^\text{th}$
generation of $x$.\smallskip\\
We define the following relations on streams: if $x,y\in X$,
\begin{itemize}
    \item $x \leq y \Leftrightarrow \forall i \in \mathbb{N},
        x_i\leq y_i$
    \item $x < y \Leftrightarrow x \leq y \land x \neq y$
    \item $x \ll y \Leftrightarrow \forall i \in \mathbb{N},
        x_i < y_i$
\end{itemize}\par

\begin{definition}[Social Welfare Relation (SWR)]
  A Social Welfare Relation is a preorder on streams. We note such a relation
$\precsim$, with $\prec$ its asymmetric part and $\sim$ its symmetric part.
\end{definition}

\begin{definition}[Social Welfare Order (SWO)]
  A Social Welfare Order (SWO) is a complete SWR.
\end{definition}

\begin{definition}[subrelation, extension]
  We define the notion of subrelation between two SWR $\precsim$ and $\precsim'$
the following way: $\precsim$ is a subrelation of $\precsim'$ iff 
$\sim\subseteq\sim'$ and $\prec\subseteq\prec'$. In this case, we also say that
$\precsim'$ is an extension of $\precsim$, and an ordering extension if
$\precsim'$ is a SWO.
\end{definition}

\begin{definition}[continuous SWR]
  A SWR $\precsim$ is continuous in some topology iff forall $x\in X$,
$\{y\in X|y\precsim x\}$ and $\{y\in X|x\precsim y\}$ are closed in that
topology.
\end{definition}

\begin{definition}[Social Welfare Function (SWF)]
  A Social Welfare Function is a function from a set of utility streams to the real numbers.
\end{definition}

\begin{definition}[(Strong) Pareto axiom for SWR]
  Let $x$ and $y$ be utility streams and $\precsim$ be a SWR.
  If $x < y$ then $x\prec y$.
\end{definition}

\begin{definition}[(Weak) Pareto axiom for SWR]
  Let $x$ and $y$ be utility streams and $\precsim$ be a SWR.
  If $x \ll y$ then $x\prec y$.
\end{definition}


\begin{definition}[(Finite) Anonymity axiom]
  Let $x$ and $y$ be utility streams and $W$ be a SWF.
  If $\exists n_0 \in\mathbb N,x_0=y_n$, $x_n=y_0$ and $\forall n\notin\{0,n_0\}, x_n=y_n$, then $W(x)=W(y)$.
\end{definition}

\begin{definition}[Pareto axiom for SWF]
  Let $x$ and $y$ be utility streams and $W$ be a SWF.
  If $x\ll y$ then $W(x)<W(y)$.
\end{definition}

\begin{definition}[Weak Pareto axiom for SWF]
  Let $x$ and $y$ be utility streams and $W$ be a SWF.
  If $\exists n_0\in\mathbb N, x_{n_0}<y_{n_0}$ and $\forall n\neq n_0, x_n=y_n$ then $W(x)<W(y)$.
  If $x\ll y$ then $W(x)< W(y)$.
\end{definition}

\begin{definition}[Dominance axiom]
  Let $x$ and $y$ be utility streams and $W$ be a SWF.
  If $\exists n_0\in\mathbb N, x_{n_0}<y_{n_0}$ and $\forall n\neq n_0, x_n=y_n$ then $W(x)<W(y)$.
  If $x\ll y$ then $W(x)\leq W(y)$.
\end{definition}


\section{Grading principle}

In this section, $Y=[0,1]$.
\smallskip\\
From \cite{svensson80}:
\begin{definition}[Grading principle $\precsim_S$]
  For all $x,y\in X$, $x \precsim_S y$ iff there exists
  $\sigma\in\mathfrak{S}_{\mathbb{N}}$ finite
  such that $x\geq\sigma y$.
\end{definition}

\begin{theorem}
  $\precsim_S$ is a paretian and anonymous quasi-ordering on $X$.
\end{theorem}

\begin{lemma}[from \cite{szpilrajn30}]
  For every preorder $\precsim$ on a set $S$, there is an order on
  $S$ compatible with $\precsim$.
\end{lemma}

\begin{theorem}
  For any extension $\precsim'$ of the grading principle 
  $\precsim_S$, there is a paretian
  and anonymous ordering extension $\precsim$ of $\precsim'$.
\end{theorem}


\section{Overtaking criterion}
In this section, $Y=[0,1]$.
\smallskip\\
From \cite{svensson80}:
\begin{definition}[Overtaking criterion $\precsim_W$]
  For all $x,y\in X$, $x \precsim_W y$ iff there exists
  $n_0\in\mathbb{N}$ such that for all $n\geq n_0$,
  $\sum_{k=1}^n (x_k - y_k)\geq 0$.
\end{definition}

\begin{theorem}
  There exist paretian and anonymous extensions of $\precsim_W$.
\end{theorem}


\section{Continuous preferences}
In this section, $Y=[0,1]$.
\smallskip\\
From \cite{diamond65}:\smallskip\par
We define two metrics:
\begin{definition}[Sup metric]
  \[d(x,y)=\sup_k |x_k - y_k|\]
\end{definition}
\begin{definition}[Product metric]
  \[d(x,y)=\sum_{k=1}^\infty 2^{-k}|x_k - y_k|\]
\end{definition}

\bigskip
From \cite{svensson80}:\smallskip\par
We consider sequences of streams $(x^n)\in X^\mathbb{N}$, and say that
$(x^n)$ converges to $x\in X$ if and only if
\[\sum_{k=1}^\infty |x^n_k - x_k| \underset{n\rightarrow\infty}{\rightarrow}0\]
Additionally, we have the following distance $d$:
\[d(x,y)=\min\left(1,\sum_{k=1}^\infty |x_k - y_k|\right)\]
which gives us a topology $T$.

\begin{theorem}
  There are continuous, paretian and anonymous SWR wrt. T.
\end{theorem}

  From \cite{diamond65}:

\begin{theorem}
  Let \(\precsim\) a SWO on \(X\). If \(\precsim\) satisfies the
  following :
  \begin{itemize}
  \item \(\precsim\) is an extension of the SWR \(\leq\)
    (\(\forall x, y \in X, x \leq y \Rightarrow x \precsim y\))
  \item \(\precsim\) satisfies the Weak Pareto Axiom
    (\(\forall x, y \in X, x \ll y \Rightarrow x \prec y\))
  \item \(\precsim\) is continuous wrt. the sup metric (resp. the
    product metric)
  \end{itemize}
  then the following holds :
  \(\forall x \in X, \exists u \in Y, x \sim u^\omega\)
\end{theorem}

\section{Existence of a social welfare function}

From \cite{basumitra03}:

\begin{theorem}
  There is no SWF satifying both Pareto and Anonymity axioms.
\end{theorem}

\begin{theorem}
  If $Y=[0,1]$, there is no SWF satifying both Dominance and Anonymity axioms.
\end{theorem}

From \cite{basumitra07p}:

\begin{theorem}
  If $Y\subset\mathbb N$, there exists a SWF satisfying both Weak Pareto and Anonymity axioms.
\end{theorem}

\bibliographystyle{alpha}
\bibliography{bibliography}
\end{document}

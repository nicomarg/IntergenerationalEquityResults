\documentclass{article}
\usepackage[utf8]{inputenc}
\usepackage[T1]{fontenc}
\usepackage{lmodern}
\usepackage[english]{babel}
\usepackage{amsmath, amssymb, amsthm, stmaryrd}

\title{Summary of the results seen in the Intergenerational Equity workshop}
\author{}

\newtheorem{definition}{Definition}
\newtheorem{theorem}{Theorem}
\newtheorem{lemma}{Lemma}

\begin{document}

\maketitle

\section{Grading principle}

From \cite{svensson80}:

\begin{definition}[Grading principle $R_S$]
    For all $x,y\in X$, $x R_S y$ iff there exists $\sigma\in\mathfrak{S}_{\mathbb{N}}$ finite
    such that $x\geq\sigma y$.
\end{definition}

\begin{theorem}
    $R_S$ is a paretian and anonymous quasi-ordering on $X$.
\end{theorem}

From Szpilrajn [TODO : citation]:
\begin{lemma}
    For every quasi-ordering $R$ in a set $S$, there is an ordering $S$ compatible with $R$.
\end{lemma}

\begin{theorem}
    To any quasi-ordering $R$ in $X$ such that the grading principle $R_S$ is a subrelation of
    $R'$, there is a paretian and anonymous order $R$ over $X$ such that $R'$ is a subrelation
    of $R$.
\end{theorem}
\bibliographystyle{alpha}
\bibliography{bibliography}
\end{document}
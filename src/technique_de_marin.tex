\documentclass{article}
\usepackage[utf8]{inputenc}
\usepackage[T1]{fontenc}
\usepackage{lmodern}
\usepackage[english]{babel}
\usepackage{amsmath, amssymb, amsthm, stmaryrd}

\newcommand{\G}{\mathbb{G}}

\newtheorem{theorem}{Theorem}

\title{Extending results for an arbitrary set of generations}
\author{}

\begin{document}

\maketitle

\section{Notations}

$\G$ is the set of generations. \\
$Y$ is the ordered set of utility values. It is not an antichain.\\
The set of utility streams is $X=Y^\G$.\\
$\kappa$ is a non-empty family of subsets of $\G$, stable by union and which doesn't
contain $\emptyset$. \\
$\mu$ is a subgroup of $\mathfrak{S}_\G$.

\section{Axioms}

We define the $\kappa$-relation as:
\[<_\kappa = \{(x,y)\in X^2|\forall i\in \G, x_i \leq y_i \land
\{j \in \G | x_j < y_j\}\in\kappa\}\]
$<_\kappa$ is antireflexive since $\emptyset\not\in\kappa$,
and transitive since $\kappa$ is stable by union. Therefore it is a strict order.
And $\leq_\kappa$ the reflexive closure of $<_\kappa$.\smallskip\par

\textbf{$\kappa$-Pareto}:
\[\forall x,y\in X, x <_\kappa y \Rightarrow x \prec y\]

\textbf{$\mu$-anonymity}:
\[\forall\sigma\in\mu,\forall x\in X, x \sim \sigma x\]

\section{Grading principle}
\ \par
\textbf{$\kappa$-$\mu$ grading principle}:
\[R_{\kappa\mu} = \{(x,y)\in X^2 | \exists\sigma\in\mu, x \leq_\kappa \sigma y\}\]

Adaptation of theorem 1 in \cite{svensson80}:
\begin{theorem}
    $R_{\kappa\mu}$ is a $\mu$-anonymous quasi-ordering on $X$, and
    $\kappa$-paretian if...
\end{theorem}
\begin{proof}
    \textbf{Reflexivity}: we can take $\sigma = Id\in\kappa$ since $\kappa$ is a
    subgroup of $\mathfrak{S}_\G$.\\
    \textbf{Transitivity}: if $x,y,z\in X$ such that $x R_{\kappa\mu} y$ and 
    $y R_{\kappa\mu} z$,
    we take $\sigma,\sigma'\in \mu$ such that $x\leq_\kappa \sigma y$ and
    $y\leq_\kappa \sigma' z$. We have $x\leq_\kappa (\sigma\circ\sigma')z$,
    and $\sigma\circ\sigma'\in\mu$ since $\sigma$ and $\sigma'$ are both in $\mu$.\\
    \textbf{$\mu$-anonymity}: if $x\in X$ and $\sigma\in\mu$, then
    $x\leq_\kappa\sigma^{-1}\sigma x$ with $\sigma^{-1}\in\mu$,
    therefore $x R_{\kappa,\mu} \sigma x$. On the other hand,
    $\sigma x \leq_{\kappa}\sigma x$, hence $x \sim_{\kappa, \mu}\sigma x$.\\
    \textbf{$\kappa$-Pareto}: if we have $x,y\in X$ with $x<_\kappa y$, is there a
    $\sigma\in \mu\mathfrak{S}_\G$ such that $y\leq_\kappa \sigma x$? 
\end{proof}

\bibliographystyle{alpha}
\bibliography{bibliography}
\end{document}
